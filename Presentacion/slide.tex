%%%%%%%%%%%%%%%%%%%%%%%%%%%%%%%%%%%%%%%%%%%%%%%%%%%%%%
% Thanks to Xu Minghao's work                        %
% I modify it into uchicago version                  %
% to not make new bug, I don't alter "Ritsumeikan"   %
% keywords in file. Pls feel free to use             %
%%%%%%%%%%%%%%%%%%%%%%%%%%%%%%%%%%%%%%%%%%%%%%%%%%%%%%

%%%%%%%%%%%%%%%%%%%%%%%%%%%%%%%%%%%%%%%%%%%%%%%%%%%%%%
% A Beamer template for Ritsumeikan University       %
% Author: Ming-Hao Xu (Xu Minghao)                   %
% Date:   April 2022.                                %
% LPPL Licensed.                                     %
%%%%%%%%%%%%%%%%%%%%%%%%%%%%%%%%%%%%%%%%%%%%%%%%%%%%%%

\documentclass[compress]{beamer}
\usepackage[utf8]{inputenc}
\usepackage[spanish]{babel}
\usepackage{amsmath}
\usepackage{amsfonts}
\usepackage{amssymb}
\usepackage{graphicx}
\usepackage{lipsum}
\usepackage{ragged2e}
\usepackage{hyperref}
\usepackage{float}
\usepackage{url}
\usepackage{multicol}
\usepackage{subfigure}

\usetheme{Berlin}
\useoutertheme{miniframes}
\setbeamertemplate{navigation symbols}{} 


\newcommand{\celda}[1]{
	\begin{minipage}{2.5cm}
		\vspace{5mm}
		#1
		\vspace{5mm}
	\end{minipage}
}

\author[Cristhian Moya Mota]{Cristhian Moya Mota}
\title[PE en Series Temporales]{Estudio sobre la Efectividad del Positional Encoding en Transformers para Series Temporales y Diseño de Mecanismos Adaptados}
\date{8 de septiembre de 2025} 

\logo{\includegraphics[scale=0.04]{pic/logo}}
\institute[UGR]{
	\inst{}
	Tutor: Julián Luengo Martín\\Departamento de Ciencias de la Computación e Inteligencia Artificial\\
	\inst{ }
	Cotutor: Diego Jesús García Gil\\Departamento de Lenguajes y Sistemas Informáticos\\
	\vspace{2mm}	
}

\AtBeginSection[]
{
	\begin{frame}<beamer>{Contenido}
		\tableofcontents[currentsection,currentsubsection]
	\end{frame}
}


\begin{document}
	
	\begin{frame}
		\maketitle
	\end{frame}
	
	\begin{frame}{Contenido}
		\tableofcontents
	\end{frame}
	
	\setbeamerfont{footnote}{size=\tiny}
	\section{Introducción}
	
	\begin{frame}{Introducción}
		\begin{figure}
			\centering
			\begin{minipage}{0.4\textwidth}
				\centering
				\includegraphics[height=\columnwidth]{pic/ts.png}
			\end{minipage}
			\hfill
			\begin{minipage}{0.4\textwidth}
				\centering
				\includegraphics[height=\columnwidth]{pic/trans.png}
			\end{minipage}
			\caption[]{Forecasting en Series Temporales\footnote{{https://developer.nvidia.com/blog/time-series-forecasting-with-the-nvidia-time-series-prediction-platform-and-triton-inference-server/}}. Transformers\footnote{
					https://doi.org/10.48550/arXiv.1706.03762}}
		\end{figure}
	\end{frame}
	
	\begin{frame}{Introducción. Justificación}
		
		Aspectos que justifican la realización del proyecto:
		\begin{itemize}
			\item Falta de captura de la estructura. Ausencia de información semántica.
			\item Dificultad para adaptarse a diferentes escalas temporales y falta de semántica.
			\item Complejidad computacional y falta de interpretabilidad en la metodología.
		\end{itemize}
	\end{frame}
	
	\section{Estado del arte}
	
	\section{Propuestas de codificación y evaluación empírica de su rendimiento}


\section{Conclusiones y trabajos futuros}
\subsection{Conclusiones}
\begin{frame}{}
		El proyecto ha conseguido:
	
\end{frame}


\subsection{Trabajos futuros}
\begin{frame}{}
	\
\end{frame}


\begin{frame}
	\centering \textbf{Gracias}

	\begin{figure}[H,font=\Small]
		\centering
		\subfigure{\includegraphics[width=0.2\textwidth]{pic/QRTFM}} 
		\label{fig:calidad}
		
		¿Preguntas?
		
	\end{figure}
	
	
\end{frame}


\end{document}