\chapter{Tendencias y Estado del arte}

Antes de adentrarnos de lleno en el análisis del problema y la realización de nuevas metodologías, es necesario estudiar y comprender el estado actual de esta temática en el estado del arte. Debido a que no se ha visto tan explotada como otros ámbitos de la IA, como es el caso de los modelos convolucionales o el aprendizaje automático supervisado tabular, podemos encontrar continuos cambios y nuevas vertientes que pueden inspirarnos a la hora de abordar el problema. Además, de manera adicional, podremos ver si algunas de nuestras ideas ya han sido implementadas y dado resultado, ya que en caso negativo, podríamos evitar comenter los mismos errores y centrarnos en una vertiente mucho más prometedora.\\

Hasta el momento, buena parte de las técnicas diseñadas especialmente para series temporales y predicción a largo plazo, se basan principalmente en la descomposición las mismas en subcomponentes mucho más sencillas de procesar, las cuales pueden seguir un enfoque similar a divide y vencerás: descomponer la red en diferentes elementos, y mediante un mecanismo de agregación (aditivo, multiplicativo, o estadístico), recomponer la solución a la tarea. Sin embargo, se han producido solo avances menores en los últimos años en dicha dirección. \\

Actualmente, la principal tendencia en el modelado de series temporales se basa en la adaptación de modelos desarrollados para otras modalidades. Por ejemplo, se han incorporado convoluciones, propias de visión, con el objetivo de capturar patrones locales. Pero, lo más frecuente es encontrar adapataciones propias de mecanismos de procesamiento de lenguaje natural, como los ya mencionados Transformers, debido a su capacidad para modelar dependencias de largo alcance de forma eficiente. Previamente, también se han explorado arquitecturas recurrentes, como las Recurrent Neural Networks (RNN) y sus variantes más avanzadas, como las Long Short-Term Memory (LSTM) \cite{6795963}, aunque estas presentaban ciertas limitaciones en cuanto a capacidad de paralelización y manejo de relaciones temporales de largo plazo.\\

A continuación, nos adentraremos de lleno en dichas propuestas, con el fin de aclarar, en primer, los conceptos clave acerca de las series temporales, y además, entender las problemáticas que surgen durante su procesado.

\section{Procecsamiento básico de Series Temporales}



\section{Modelos basados en descomposición}

\section{Transformers y Positional Encoding}

\section{Conjuntos de datos disponibles}
