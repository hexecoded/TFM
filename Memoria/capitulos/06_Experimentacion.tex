\chapter{Análisis y estudio de los resultados obtenidos}

Una vez expuestas todas las condiciones del entrenamiento, como los modelos a probar, las métricas, funciones de pérdida y presentados los datos, podemos dar paso al entrenamiento. Sin lugar a dudas, es la fase más lenta de todo el proceso, la cual ha prolongado su desarrollo considerablemente. El objetivo es claro: poner a prueba las 5 propuestas realizadas para establecer, en primer lugar, cuál o cuáles de ellas son competitivas, y posteriormente, evaluar todos los conjuntos de datos para medir la ganancia con respecto al conjunto de datos original.

\section{Consideraciones previas: ¿Cómo detectar si un modelo es mejor de manera justa?}

A la hora de realizar las pruebas, existen multitud de factores que pueden influir en los resultados, y los cuales debemos controlar minuciosamente para evitar problemas que alteren los resultados y nos den una idea equivocada del rendimiento. Entre ellos, podemos encontrar las inicializaciones aleatorias de los pesos, el riesgo de sobreaprendizaje y en casos de extremada similaridad de resultados, la calidad semántica aportada por la codificación que estamos evaluando.\\

Para tratar de minimizar el impacto de estos factores y facilitar el análisis de calidad de los positional encodingds propuestos, tomaremos una serie de medidas las cuales examinaremos a continuación.

\subsection{Impacto de la inicialización aleatoria}

Cuando entrenamos cualquier modelo de aprendizaje, en prácticamente cualquier ámbito de problema (datos tabulares, imágenes, secuencias,..) existe un paso el cual muchas veces es ignorado, y que puede afectar considerablemente a los resultados obtenidos: la inicialización de los parámetros del modelo. Aunque este paso podría parecer trivial y de escasa relevancia, debido a que los parámetros serán posteriormente ajustados mediante los mecanismos de aprendizaje, en realidad, puede desempeñar un papel determinante en el rendimiento final del modelo.\\

En la práctica, la inicialización suele realizarse a partir de una distribución pseudoaleatoria generada mediante secuencias controladas por el sistema operativo o el propio lenguaje de programación. Este proceso asigna valores iniciales a cada una de las conexiones y pesos del modelo, y cada valor asignado puede influir considerablemente en la velocidad de convergencia y la estabilidad numérica, ya que los optimizadores que ajustan sus valores posteriormente, se basan en métodos de descenso de gradiente o búsqueda local para tratar de encontrar óptimos locales.\\

Para mejorar la convergencia y eficiencia en entrenamiento, se han desarrollado numerosos optimizadores con el objetivo de evitar quedar atascado durante este proceso y tratar de escapar de óptimos locales de baja calidad que pudieran afectar al resultado final. Uno de los más empleados es Adam~\cite{kingma2017adammethodstochasticoptimization}, el cual utiliza momentum para dotar de cierta inercia a cada una de las dimensiones, ajustando una a una su tasa de aprendizaje, y así tratar de escapar de localidades subóptimas.\\

Por ello, emplearemos este optimizador en el entrenamiento del Transformer, ya que nos permite un enfoque más estable y adaptativo en la actualización de los parámetros y reduciendo la sensibilidad a los pesos y la tasa de aprendizaje inicial.\\
 
Pero esta estrategia no nos exime de los problemas derivados de malas inicializaciones que, de forma fortuita, podrían producir un mal resultado en un modelo que, en condiciones normales, tendría un desempeño superior. Dicho fenómeno puede verse potenciado al emplear los diferentes encodings ponderados, ya que una mala inicialización de alguno de los pesos podría favorecer artificialmente una de las variantes desde el inicio. Aunque el encoding asociado no proporcione un buen rendimiento, la reducción de su valor podría ser lenta, sentenciando el experimento desde el inicio. Para evitarlo, aplicaremos las siguientes soluciones:

\begin{itemize}
	\item Ejecutar todos los experimentos múltiples veces utilizando diferentes semillas aleatorias. Esta práctica nos permitirá reducir el impacto de la aleatoriedad en los modelos, asegurando que los resultados reflejen de manera más fiel el comportamiento real del modelo.
	\item Realizar un inicio uniforme de los pesos de ponderación de encodings. Esto se traduce en establecer como valor de partida a cada peso aprendible un valor equivalente a $1/n$, siendo $n$ el número de alternativas que se desea evaluar. De esta forma, el cálculo empieza sin condiciones de dominancia, y será el propio proceso de aprendizaje el que altere los pesos a su beneficio.
\end{itemize}

 
Gracias a estas soluciones, paliaremos en gran medida el impacto en los resultados. No obstante, debemos tener cuidado con el número de ejecuciones de cada modelo, pues cada una puede requerir varios días de cómputo, lo que nos obliga a buscar un equilibrio entre estabilidad y tiempo. Por lo tanto, excepto si se indica lo contrario, se realizarán tres ejecuciones independientes para cada configuración experimental, calculando posteriormente el promedio estadístico de cada una de las métricas. De esta manera, logramos un \textit{trade-off} adecuado entre ambas condiciones. En conjunto con el uso de Adam, trataremos de maximizar la estabilidad y confiabilidad de los resultados.
 
\subsection{Sobreaprendizaje}

Otro de los problemas destacados durante en el entrenamiento puede ser el sobreaprendizaje. Cuando usamos arquitecturas tan profundas y potentes como los Transformers, es sencillo que la arquitectura tome el conjunto de entrenamiento y lo aprenda de manera bastante próxima a la distribución de estos datos, es decir, la acabe ``memorizando'' y la capacidad de generalización del modelo sea pobre. Esto es especialmente notable en conjuntos de datos más sencillos o de menor longitud, en los que el modelo podría incluso verse algo sobredimensionado. Para evitarlo. se han seleccionado dos estrategias: regularización mediante \textit{dropout} y uso de \textit{early stopping}.

\begin{itemize}
	\item Con el \textbf{early-stopping}, estamos añadiendo un mecanismo de parada automática del entrenamiento, el cual, especificando una paciencia, en épocas, detendremos el modelo en ausencia de mejora. Para controlar esta parada, monitorizaremos el error de pérdida en validación. Para mantener tener una paciencia algo tolerante, escogeremos como valor 3 épocas; si en 3 épocas el error del modelo ha sido peor que el mejor resultado registrado, cancelaremos el proceso, ahorrando recursos de cómputo y evitando mala generalización.
	\item Mediante \textbf{dropout}, evitaremos que la capas totalmente conectadas de la red se sobreespecialicen, desactivando aleatoriamente algunas neuronas en las diferentes iteraciones del entrenamiento. Por defecto, Informer ya incluye un parámetro de dropout, cuyo valor inicial es bastante bajo, 0.05. Este dropout se aplica no solo a las capas del encoder y decoder, sino también a la estructura del embedding de entrada cuando este contiene componentes aprendibles. Dado que ambos elementos presentan comportamientos distintos, y que regularizar los datos de manera excesiva no es conveniente, pues podría perderse información, se ha establecido como criterio aplicar un dropout mayor, de 0.2, para las capas del modelo, mientras que se mantiene el dropout del embedding en su valor original de 0.05. Así, de alguna forma estamos especificando una regla heurística, en la cual mantenemos el índice asociado al embedding 4 veces menor que el aplicado en las capas de la arquitectura.
\end{itemize}

Con el uso de estos dos factores, podremos minimizar el impacto del \textit{overfitting}, y evitar una mala capacidad de generalización del modelo.

\subsection{Comparando la calidad semántica del encoding. Técnica del barajado}

En los aparatados anteriores, hemos destacado constantemente la importancia semántica del positional encoding, y en la importancia de medir adecuadamente la calidad de las codificaciones a través de este aspecto. Sin embargo, la componente semántica es algo completamente externo para una arquitectura de este tipo, ya que las capas de esta solo entienden de valores numéricos y tensores, pero no es capaz de detectar información de significado. Sin embargo, esta puede encontrarse de manera implícita en los datos, en aspectos como la localidad y la detección de patrones, tareas las cuales hemos tratado de modelar en las propuestas de codificación formuladas.\\

Existe una manera muy sencilla de verificar esta propiedad y evaluar la utilidad de la codificación añadida: mezclar la entrada del decoder durante la inferencia y analizar su impacto en la calidad del resultado final, en comparación con una entrada intacta, sin desordenar. Este procedimiento fue propuesto por la publicación \textit{``Are Transformers Effective for Time Series Forecasting?''}~\cite{zeng2022transformerseffectivetimeseries}, y nos permite evidenciar de manera intuitiva la presencia o ausencia de orden y localidad en la codificación.\\

De manera inherente a su estructura, el mecanismo de self-attention de los Transformers es invariante al orden de los elementos de la secuencia. Cuando realizamos el cálculo de las matrices de atención, no estamos realizando más que un producto cruzado de todos los tokens con los demás, por lo que el orden de los productos no afectaría al resultado. Es por ello que se necesita un encoding posicional que permita incluir dicha relación de orden en los datos, y se cumpla esta propiedad esencial necesaria en las series temporales.\\

Sin embargo, en el artículo mencionado se llega a la conclusión de que los mecanismos existentes para los principales modelos del estado del arte, como Informer, Autoformer y FEDformer, con sus respectivas codificaciones temporales, no son capaces de captar el orden natural de los datos. Y va incluso más allá: sustituir todo el mecanismo de atención por una capa totalmente conectada es capaz de ofrecer mejores resultados, y al aplicar el barajado, sí se ve afectada en mayor medida por el desorden. \\

En nuestro caso, teniendo en cuenta que nuestra arquitectura se basa en la estructura de Informer, pero simplificando el mecanismo de atención al prescindir del muestreo y las convoluciones, la escasez mantenimiento del orden podría afectarnos de manera similar. Pero, en este caso, queremos evaluar la efectividad de los métodos de codificación propuestos en este trabajo, por lo que repetiremos el experimento, pero en lugar de hacerlo con la codificación por defecto, que ya conocemos sus resultados, con nuestras alternativas. El objetivo es claro: tratar de demostrar que permiten conservar una mayor semántica de los datos y se ven más perjudicados frente al desorden que la tradicional codificación seno-coseno.

\section{Resultados de los experimentos}

Una vez justificado el procedimiento de ejecución a seguir para los experimentos realizados, podemos comenzar a analizar los resultados obtenidos en el entrenamiento de los mismos. Tal y como se especificó en el capítulo \ref{cap5}, compararemos los resultados haciendo uso de las métricas de regresión especificadas, pero haciendo especial hincapié en el MAE y MSE.\\

La estructura seguida por los experimentos será la siguiente: primero, evaluaremos cuales de las cinco propuestas formuladas obtiene el mejor rendimiento sobre un conjunto de referencia, Household Power Consumption~\cite{hebrail2006individual}, y posteriormente, procederemos a analizar su utilidad en el resto de conjuntos de datos.

\subsection{Evaluación de las alternativas sobre el conjunto HPC}

El dataset de Individual Household Power Consumption ha sido el conjunto escogido para evaluar, en primer lugar, las 5 variantes de nuestro método de encoding. Esta decisión viene motivada por dos aspectos; en primer lugar, este conjunto de datos dispone de una longitud suficiente para inferir la información relevante de la serie y realizar predicciones a largo plazo, y además, en el comportamiento de sus variables vimos aspectos como la estacionalidad, la cual es más difícil de modelar para un Transformer, y ausencia de estacionalidad, lo cual a priori podría suponer un problema para la aplicación de métodos clásicos. Pero, principalmente, la decisión de escoger un dataset viene motivada por le coste computacional: evaluar todas las alternativas en cada uno de los conjuntos de datos podría requerir semanas e incluso meses para completar por cada dataset, lo cual se escapa de los recursos y limitaciones temporales de las que disponemos.\\

\subsubsection{Resultados}

Por lo tanto, escoger un conjunto de datos más manejable como este, de aproximadamente 2 millones de mediciones, y 7 características, supone un interesante campo de pruebas para las codificaciones. Para la comparativa, realizaremos las siguientes ejecuciones:

\begin{itemize}
	\item Informer con configuración por defecto.
	\item PE sinusoidal.
	\item Nuevas propuestas: WinStat, WinStatLag, WinStatFlex,  WinStatTPE y WinStatSPE	
	\item Sin PE.
\end{itemize}

Para realizar una comparativa justa de los resultados, se han empleado en todos los casos los mismos parámetros, de forma que criterios como la longitud de secuencia, la longitud de contexto o la longitud de predicción no afecten a los resultados. La arquitectura y los parámetros asociados se pueden ver en la tabla \ref{ajustes}.

\begin{table}[!ht]
	\centering
	\begin{tabular}{l|l}
		\toprule
		Parámetro & Valor \\
		\midrule
		{Modelo de atención} & Vanilla \\
		{Tamaño de batch} & 32 \\
		{Dropout} & 0.2 \\
		{Nº de ejecuciones} & 3 \\
		{Tamaño de ventana} & 60 \\
		{Longitud de secuencia} & 180 \\
		{Longitud de etiqueta/contexto} & 60 \\
		{Longitud de predicción} & 60 \\
		\bottomrule
	\end{tabular}
\caption{HPC: configuración de parámetros comunes empleada para los experimentos}
\label{ajustes}
\end{table}

\paragraph{Informer}

En primer lugar, comenzaremos por ejecutar el modelo original de Informer. De esta forma, tendremos una referencia de partida que nos sirva para comparar las propuestas siguientes, y ver si están siendo efectivas. Por defecto, Informer usa codificación de tipo seno-coseno, pero adicionalmente, incorpora información de los timestamps a través de un mecanismo llamado timeF. Son características adicionales que tratan de ayudar al modelo a capturar patrones periódicos y estacionales en la serie temporal, aunque de manera bastante limitada. 

\begin{table}[!ht]
	\centering
\begin{tabular}{lcc}
	\toprule
	Métrica & Media & STD \\
	\midrule
	MSE & 0,532965 & 0,010976 \\
	MAE & 0,415272 & 0,012216 \\
	MAPE & 2,598725 & 0,066858 \\
	MSPE & 1345,903158 & 113,73071 \\
	RMSE & 0,730005 & 0,00754 \\
	TrainTime(s) & 7817,02026 & 821,270911 \\
	TestTime(s) & 91,941244 & 1.244423 \\
	\bottomrule
\end{tabular}
\caption{HPC: resultados para Informer}
\label{hpcinf}
\end{table}

Los resultados de la ejecución pueden apreciarse en la tabla \ref{hpcinf}. En ella, podemos notar que la varianza en los resultados, tras ejecutar 3 iteraciones, es bastante pequeña, e inferior a la centésima tanto para MAE como MSE. Sin embargo, las tasas de error en las métricas de MAPE y MSPE se disparan: esto se debe a su gran sensibilidad a valores pequeños, ya que todas las mediciones han sido normalizadas, por lo que no deberían ser consideradas como una forma de comparación debida su inestabilidad. En el caso de RMSE, el error también es bastante alto, de un 73\%, fruto de la misma debilidad que la mostrada en MAPE y MSPE.\\

En términos de eficiencia computacional, podemos utilizar el tiempo medio de ejecución como referencia. En promedio, una ejecución del modelo por defecto requiere aproximadamente 7817 segundos, equivalentes a casi 2,17 horas. Aunque este tiempo resulta razonable considerando la longitud del conjunto de datos, su utilidad real es servir como valor de referencia para evaluar el coste adicional que implican las operaciones introducidas en los modelos alternativos que veremos a continuación.

\paragraph{PE sinusoidal}

Continuando con el modelo de referencia, daremos un segundo paso. Teniendo en cuenta el impacto limitado de la información añadida por timeF, ¿es realmente relevante frente a la información proporcionada por el encoding de tipo seno-coseno?. Para conocerlo, hemos realizado una ejecución suprimiendo esta componente, convirtiendo la arquitectura en prácticamente la definición de un Transformer vanilla para procesamiento de lenguaje. 


\begin{table}[!ht]
	\centering
	\begin{tabular}{lcc}
		\toprule
		Métrica & Media & STD \\
		\midrule
		MSE & 0,537721 & 0,011375 \\
		MAE & 0,414084 & 0,006756 \\
		MAPE & 2,573043 & 0,094281 \\
		MSPE & 1363,108439 & 117,15614 \\
		RMSE & 0,733254 & 0,007743 \\
		TrainTime(s) & 10741,03553 & 422,837205 \\
		TestTime(s) & 225,145923 & 63,83665 \\
		\bottomrule
	\end{tabular}
	\caption{HPC: resultados para encoding seno-coseno}
	\label{hpcfixed}
\end{table}

Sorprendentemente, los resultados obtenidos, visibles en la tabla \ref{hpcfixed} son muy similares a los vistos en el modelo original: la pérdida de rendimiento en métricas es prácticamente despreciable, con una diferencia de rendimiento inferior a la centésima para el MSE (0,5329 vs 0,5377) y un caso similar en MAE (0,4152 vs 0,414), obteniendo incluso un mejor el resultado para esta codificación. Esto nos indica que la información adicional aportada en el modelo original no estaba aumentando considerablemente el rendimiento del modelo.\\

Sin embargo, resulta relevante destacar que el tiempo de ejecución aumenta significativamente, alcanzando aproximadamente 10741 segundos (cerca de 3 horas) en el entrenamiento. Aunque este resultado pueda parecer contraintuitivo, ya que eliminar componentes debería reducir el coste, al emplear esta variante el proceso de convergencia del modelo se desarrolla de forma más lenta, y provoca un aumento del tiempo de ejecución. En defintiva, suprimir la información contextual adicional de Informer no empeora gravemente los resultados, pero disminuye la velocidad de convergencia durante el aprendizaje.\\

A continuación, comenzaremos a probar las nuevas alternativas, para ver si consiguen mejorar estos resultados.

\paragraph{WinStat}

Comenzamos por la primera de las propuestas, la cual se basa únicamente en la adición información estadística a través de una ventana deslizante.\\

Recordando su funcionamiento, este método requiere especificar el tamaño de la ventana, la cual recorre los datos para extraer y concatenar información local sobre la media, la desviación estándar y los valores extremos en su entorno. Por tanto, la elección de un tamaño de ventana adecuado resulta crítico para garantizar un rendimiento óptimo, sin comprometer la eficiencia del modelo. Como criterio heurístico, se empleará una longitud comprendida entre un tercio y un cuarto del tamaño de la secuencia de entrada, ya que ofrece un equilibrio razonable entre el mínimo posible, 3, y el extremo opuesto de utilizar una ventana de tamaño longitud de la secuencia.

\begin{table}[!ht]
	\centering
	\begin{tabular}{lcc}
		\toprule
		Métrica & Media & STD \\
		\midrule
		MSE & 0,502033 & 0,004632 \\
		MAE & 0,394002 & 0,002849 \\
		MAPE & 2,487715 & 0,090071 \\
		MSPE & 1196,131307 & 105,107660 \\
		RMSE & 0,708535 & 0,003274 \\
		TrainTime(s) & 17697,790107 & 3174,41485 \\
		TestTime(s) & 253,980487 & 3,765932 \\
		\bottomrule
	\end{tabular}
	\caption{HPC: resultados para encoding WinStat}
	\label{hpcwin}
\end{table}

Los resultados para esta alternativa son significativamente mejores y más estables (tabla \ref{hpcwin}) que en los casos anteriores, ya que hemos conseguido reducir tanto MAE como MSE, reduciendo además la varianza de los resultados. Tomando esta última como referencia, podemos apreciar que la mejora es superior a las tres centésimas (0,502 vs 0,537), aportando una mejora considerable del 6\%. Sin embargo, esta mejora de rendimiento ha tenido un gran impacto en el tiempo de cómputo: este es ahora mucho mayor que en casos anteriores, con una variabilidad además mucho más alta, de casi una hora.\\

En promedio, se ha requerido 17697 segundos, unas 4,9 horas, para completar cada ejecución, lo cual es aumento del tiempo de ejecución a prácticamente el doble de Informer original. Si bien la mejora ha sido notable, sería deseable observar si alguna de las siguientes alternativas consigue mejorar la estabilidad numérica y la convergencia de los resultados.

\paragraph{WinStatLags}

WinStatLags es la siguiente propuesta de encoding a evaluar, siguiendo la estrategia especificada de ir aumentando de manera creciente la complejidad de los modelos. La principal característica de este modelo consistía en la incorporación de diferencias mediante lags, las cuales potencian el contexto local. Siguiendo las misma configuración que en los casos anteriores, obtenemos los resultados de la tabla \ref{hpcwinlags}:


\begin{table}[!ht]
	\centering
	\begin{tabular}{lcc}
		\toprule
		Métrica & Media & STD \\
		\midrule
		MSE & 0.491026 & 0.004988 \\
		MAE & 0.398612 & 0.005025 \\
		MAPE & 2.312104 & 0.086223 \\
		MSPE & 1012.281759 & 108.017980 \\
		RMSE & 0.700723 & 0.003566 \\
		TrainTime(s) & 13071.568639 & 1403.397188 \\
		TestTime(s) & 229.312370 & 3.086725 \\
		\bottomrule
	\end{tabular}
	\caption{HPC: resultados para encoding WinStatLags}
	\label{hpcwinlags}
\end{table}


