\chapter{Introducción}

En este primer punto, describiremos de forma breve la motivación a realizar este TFM. Analizaremos la importancia de poseer herramientas capaces de realizar una predicción a largo plazo en series temporales (en inglés,\textit{ Long-term Time Series Forecasting},  LSTF), pero, sobre todo, centrándonos en un aspecto clave: la codificación posicional (\textit{Positional Encoding}, en adelante, PE), común a prácticamente todas las alternativas actuales del estado del arte. 

Procederemos a justificar su importancia, así como a formular los objetivos que se cubren con la realización de este trabajo.
\section{Motivación}

En la actualidad, los datos son uno de los bienes más preciados. Las telecomunicaciones nos han permitido alcanzar un volumen inimaginable de información digital, la cual no somos prácticamente capaces de procesar, y extraer conocimiento útil se convierte en una tarea complicada. Su gran variedad y modalidad hace necesario disponer de modelos multimodales cada vez más complejos para procesarlos, como los modelos fundacionales, para tratar así de disponer de una herramienta cercana a ser capaz de procesar todo tipo de información.\\

Hemos podido apreciar grandes avances en el procesamiento de texto, con los grandes modelos de lenguaje como GPT, el cual se ha convertido en una herramienta que usamos habitualmente para resolver nuestras dudas. Ahora, incluso permite generar imágenes, video y audio, y pensar profudamente las respuestas.\\

Sin embargo, estos no son los únicos tipos de datos que podemos emplear para aprender. Los datos basados en flujos, y las series temporales, son un recurso clave que podemos emplear para resolver multitud de problemáticas. Podemos predecir a largo plazo, clasificar fenómenos, o bien, incluso detectar anomalías. En este trabajo, nos centraremos sobre todo en la primera tarea y las dificultades que existen en este ámbito.

\subsection{Uso de Series Temporales}
\subsection{La dificultad de predicción a largo plazo: información posicional}
\section{Justificación}


\section{Objetivos}

\section{Planificación}